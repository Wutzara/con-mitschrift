\section{Einstieg in Streams (mit und ohne
Parallelisierung)}\label{einstieg-in-streams-mit-und-ohne-parallelisierung}

Um aus einem Integer-Array einen Stream zu generieren, muss man über die
\textbf{Arrays} Klasse gehen

\begin{minted}{java}
int[] zahlen = {1, 2, 3, 4};
IntStream is = Arrays.stream(zahlen);
\end{minted}

\subsection{Aufgabe 1 - IntStream}\label{aufgabe-1---intstream}

Gegeben
\texttt{int{[}{]}\ zahlen\ =\ \{3,\ 5,\ 1,\ 3,\ 7,\ 29,\ 33,\ 49,\ 5,\ 1,\ 1,\ 2,\ 3\};}

\subsubsection{Summe IntStream}\label{summe-intstream}

\begin{minted}{java}
IntStream is = Arrays.stream(zahlen);
System.out.Println(is.sum());
\end{minted}

\begin{minted}{text}
Output:
142
\end{minted}

\subsubsection{Durschnitt IntStream}\label{durschnitt-intstream}

\begin{minted}{java}
IntStream is = Arrays.stream(zahlen);
System.out.Println(is.average().getAsDouble());
\end{minted}

\begin{minted}{text}
Output:
10.923076923076923
\end{minted}

\subsubsection{Maximum IntStream}\label{maximum-intstream}

\begin{minted}{java}
IntStream is = Arrays.stream(zahlen);
System.out.Println(is.max());
\end{minted}

\begin{minted}{text}
Output:
49
\end{minted}

\subsubsection{Histogramm IntStream}\label{histogramm-intstream}

\begin{minted}{java}
IntStream s = Arrays.stream(zahlen);
Map<Integer, Long> histogramm = 
    s.boxed().collect(Collectors.groupingBy(
        Integer::intValue, Collectors.counting()));
System.out.println(histogramm);
\end{minted}

\begin{minted}{text}
Output:
{49=1, 33=1, 1=3, 2=1, 3=3, 5=2, 7=1, 29=1}
\end{minted}

\subsubsection{Ausgabe in der Form {[}3, 5, 3, \ldots{}, 2,
3{]}}\label{ausgabe-in-der-form-3-5-3-2-3}

\begin{minted}{java}
IntStream s = Arrays.stream(zahlen);
String str = "["
             + s.mapToObj(i -> String.valueOf(i)).collect(Collectors.joining(", "))
             + "]";
System.out.println(str);
\end{minted}

\begin{minted}{text}
Output:
[3, 5, 1, 3, 7, 29, 33, 49, 5, 1, 1, 2, 3]
\end{minted}

\subsection{Aufgabe 2 - Stream}\label{aufgabe-2---stream}

\subsubsection{Summe}\label{summe}

\begin{minted}{java}
List<Integer> zahlen = 
    Arrays.asList(new Integer[] { 3, 5, 1, 3, 7, 29, 33, 49, 5, 1, 1, 2, 3 });
System.out.println(zahlen.stream()
                  .collect(Collectors.summingInt(Integer::intValue)));
\end{minted}

\subsubsection{Durchschnitt}\label{durchschnitt}

\begin{minted}{java}
List<Integer> zahlen = 
    Arrays.asList(new Integer[] { 3, 5, 1, 3, 7, 29, 33, 49, 5, 1, 1, 2, 3 });
System.out.println(zahlen.stream()
                  .collect(Collectors.averagingInt(Integer::intValue)));
\end{minted}

\subsubsection{Maximum}\label{maximum}

\begin{minted}{java}
List<Integer> zahlen =
    Arrays.asList(new Integer[] { 3, 5, 1, 3, 7, 29, 33, 49, 5, 1, 1, 2, 3 });
System.out.println(zahlen.stream()
                  .max(Integer::compareTo).get());
\end{minted}

Achtung! \texttt{stream().max(...)} liefert ein
\texttt{Optional\textless{}Integer\textgreater{}} weswegen man
eigentlich eine Fallunterscheidung wie im folgenden Beispiel machen
müsste (aus Einfachheitsgründen verzichte ich darauf)

\begin{minted}{java}
List<Integer> zahlen =
    Arrays.asList(new Integer[] { 3, 5, 1, 3, 7, 29, 33, 49, 5, 1, 1, 2, 3 });
Optional<Integer> optInt = zahlen.stream().max(Integer::compareTo);
if(optInt.isPresent()) System.out.println(optInt.get());
\end{minted}

\subsubsection{Histogramm}\label{histogramm}

\begin{minted}{java}
List<Integer> zahlen =
    Arrays.asList(new Integer[] { 3, 5, 1, 3, 7, 29, 33, 49, 5, 1, 1, 2, 3 });
System.out.println(zahlen.stream()
                  .collect(Collectors
                  .groupingBy(Integer::intValue, Collectors.counting())));
\end{minted}

\subsubsection{Ausgabe}\label{ausgabe}

\begin{minted}{java}
List<Integer> zahlen =
    Arrays.asList(new Integer[] { 3, 5, 1, 3, 7, 29, 33, 49, 5, 1, 1, 2, 3 });
System.out.println("[" 
                    + zahlen.stream()
                      .map(i -> i.toString())
                      .collect(Collectors.joining(", "))
                    + "]");
\end{minted}

\section{Parallelisieren mit Streams}\label{parallelisieren-mit-streams}

Gegeben ist die folgende recht ineffiziente Primzahl-Testmethode für
ganze Zahlen \textgreater{}= 2. Diese Methode ist ein Prädikat für ganze
Zahlen im Sinne von Java 8.

\begin{minted}{java}
static boolean isPrime(int zahl) {
    int teiler = 2;
    while (zahl % teiler != 0)
        teiler++;
    if (zahl == teiler)
        return true;
    return false;
}
\end{minted}

\subsection{Aufgabe 1}\label{aufgabe-1}

Geben Sie die Primzahlen im Bereich 1..1000 aus

\begin{minted}{java}
System.out.println(IntStream
            .rangeClosed(1, 1000)
            .filter(MyIntStream::isPrime)
            .mapToObj(i -> String.valueOf(i))
            .collect(Collectors.joining(", ")));
\end{minted}

\subsection{Aufgabe 2}\label{aufgabe-2}

Geben Sie die Primzahlen im Bereich 1..100000 aus. Nützen Sie alle Kerne
Ihres Rechners zur Berechnung!

\begin{minted}{java}
System.out.println(IntStream
            .rangeClosed(1, 100000)
            .parallel()
            .filter(MyIntStream::isPrime)
            .mapToObj(i -> String.valueOf(i))
            .collect(Collectors.joining(", ")));
\end{minted}

\subsection{Aufgabe 3 (Optional)}\label{aufgabe-3-optional}

Wenn Ihnen die obige Methode zur Ermittlung von Primzahlen zu primitiv
ist, geben Sie eine effizientere Variante an.

\section{Ein einfacher Spliterator}\label{ein-einfacher-spliterator}

In einer der Aufgaben werden \mintinline{java}{Path}-Objekte (d. h.
Objekte einer Klasse, die die \texttt{Path}-Schnittstelle
implementieren) von einem Erzeuger in eine \texttt{BlockingQueue}
gestellt.

\subsection{Aufgabe 1: einfache Erzeuger-Verbraucher-Kopplung über einen
Stream}\label{aufgabe-1-einfache-erzeuger-verbraucher-kopplung-uxfcber-einen-stream}

Eine \texttt{BlockingQueue} ist eine \texttt{Collection}. Damit erhalten
wir einen Stream im Sinne von Java 8 zum Durchlaufen der Resultate des
o. a. Erzeugers. Verwenden Sie diesen Stream in einem bzw. zwei
Verbraucher(n) zum Auslesen aller \texttt{Path}-Objekte.

Starten Sie einen Erzeuger und 1-2 Verbraucher. Das Hauptprogramm soll
sich erst beenden, wenn sich alle Erzeuger bzw. Verbraucher beendet
haben.

Wie verhält sich ihr Programm?

\subsection{Aufgabe 2: ein selbst erstellter
Spliterator}\label{aufgabe-2-ein-selbst-erstellter-spliterator}

Geben Sie einen \texttt{Spliterator} für die o. a.
\texttt{BlockingQueue} an. Sie können hierzu die Methode \texttt{stream}
der Klasse \texttt{java.util.stream.StreamSupport} benützen. Stellen Sie
sicher, dass alle vom Erzeuger gelieferten Objekte auch vom Verbraucher
gelesen werden.

\section{Aufgaben aus Java Magazine von
Oracle}\label{aufgaben-aus-java-magazine-von-oracle}

\begin{minted}{java}
import java.util.Random;
import java.util.stream.Collectors;
import java.util.stream.IntStream;

public class JavaMagazine2014Sept {
    static final int MAXSEEDVALUE = 200_000;
    static final int SEEDVALUE = new Random().nextInt(MAXSEEDVALUE);
    static final int COUNT = 10;

    public static void main(String[] args) {
        System.out.println(IntStream.rangeClosed(SEEDVALUE+1,MAXSEEDVALUE)
        .parallel()
        .filter(i -> IntStream. range (2, i)
        .filter(j-> i%j == 0)
        .count() == 0)
        .limit(COUNT)
        .mapToObj(String::valueOf )
        .collect(Collectors.joining("")));
    }
}
\end{minted}

This program finds the \texttt{COUNT} number of prime numbers that are
greater than some random starting value. It runs slowly because its not
making effective use of the Stream API. What change could be made that
would result in a considerable speedup?

\begin{enumerate}
\def\labelenumi{\arabic{enumi}.}
\tightlist
\item
  Replace the lambda expression
  \texttt{(j\ -\textgreater{}\ i\ \%\ j\ ==\ 0)} by an anonymous class
  implementing the \texttt{IntPredicate} interface
\item
  Move the \texttt{limit()} function before the outer \texttt{filter()}
\item
  Use \texttt{noneMatch()} instead of the inner \texttt{filter()}
\item
  Replace both \texttt{filter()} functions using the iteration of the
  Java Collections Framework.
\end{enumerate}

\section{Desktop-Suche}\label{desktop-suche}

Gegeben ist ein GUI Rahmenprogramm (siehe con-ws14-blatt07.zip im
eLearning)

\subsection{Aufgabe 1}\label{aufgabe-1-1}

Auf Drückn der \texttt{Start}-Schaltfläche soll eine nebenläufige Suche
gestartet werden. Teilergebnisse und das Endergebnis der Suche sollen
angezeigt werden (im Hauptfenster). Benützen Sie dazu eine Ihrer
Lösungen der Aufgabe 4.

\subsection{Aufgabe 2}\label{aufgabe-2-1}

Nach dem Drücken des \texttt{Stop}-Schalters sollen alle gestarteten
Aktivitäten so bald wie möglich beendet werden. Die laufende GUI soll
nicht abgebrochen werden. \texttt{System.exit()} löst das Problem nicht.

\section{Optimistisches Sperren}\label{optimistisches-sperren}

\subsection{Aufgabe 1 - Sieht leicht aus, ist aber nicht ganz
leicht}\label{aufgabe-1---sieht-leicht-aus-ist-aber-nicht-ganz-leicht}

Versuchen Sie, einen Zähler in einem System mit Nebenläufigkeit für
ganze Zahlen zu implementieren. \texttt{next()} soll beginnend ab 0
fortlaufend ganze Zahlen liefern. \emph{Die Performance sollte besser
sein, als bei einer Lösung mit einer
\texttt{synchronized\ next()}-Funktion.}

\begin{minted}{java}
class Counter {
    private ... int counter;
    public int next() { return counter++; } // Liefern und schalten
}
\end{minted}

\begin{important}
Prinzipiell wird man eine Lösung bei den sog. optimistischen Sperren suchen.
\textbf{Aber}: man kann wohl nur dann Lösungen mit kürzeren Antwortzeiten
finden, wenn man die Phasen der Sperrung reduziert. Einen möglichen Ansatz
zeigt D. Lea mit seinem Programm `Spinlock` (siehe unten).
\end{important}

\subsection{Optimistische Updates: {[}Lea99,
2.4.4.2{]}}\label{optimistische-updates-lea99-2.4.4.2}

Ein optimistisches Update hat 3 Phasen

\begin{enumerate}
\def\labelenumi{\arabic{enumi}.}
\tightlist
\item
  Innerhalb einer Sperre: Kopieren des aktuellen Zustands
\item
  Außerhalb der Sperre: Kopie ändern
\item
  Innerhalb einer Sperre: Zurückschreiben, aber nur, falls sich der
  Zustand nicht geändert hat
\end{enumerate}

Optimistische Update-Techniken Sperren den Zugriff auf Objekte nur kurz,
insbesondere bei Multicore-Prozessoren.

\begin{minted}{java}
class Optimistic { // Generic code sketch
    private State state;

    private synchronized State getState() {
        return state;
    }

    private synchronized boolean commit(State assumed, State next) {
        if (state == assumed) {
            state = next;
            return true;
        } else {
            return false;
        }
    }
}
\end{minted}

\subsubsection{Was tun, wenn der commit nicht
klappt?}\label{was-tun-wenn-der-commit-nicht-klappt}

Ansatz: wir probieren es solange, bis es klappt. Dazu wollen wir die
Sperre möglichst kurz halten, wie im folgenden Beispiel für einen
Spinlock aus {[}Lea99, 3.2.6.5{]}. Wenn Sie das Beispiel selbst
implementieren, werden Sie sehen, dass die Laufzeiten steigen, wenn man
nach den Fehlschlägen beim \texttt{commit} sofort wieder probiert, ob
das \texttt{commit} erfolgreich ist. Die Schwierigkeit besteht darin,
die \emph{richtigen} Pausen für die Wiederholung zu finden.

\subsubsection{Aktives Warten: Busy Wait nach {[}Lea99,
3.2.6.5{]}}\label{aktives-warten-busy-wait-nach-lea99-3.2.6.5}

\begin{important}
Busy Wait ist nur für extreme Sonderfälle sinnvoll. Meist funktioniert die
einfache Lösung mit \mintinline{java}{wait()} und
\mintinline{java}{notifyAll()} besser. Zuverlässiger ist sie in jedem Fall.
\end{important}

Das folgende Listing zeigt die einfachste Form eines sog. ``Spinlock''.

\begin{minted}{java}
protected void busyWaitUntilCond() {
    while(!busy) {
        Thread.yield();
    }
}
\end{minted}

\begin{enumerate}
\def\labelenumi{\arabic{enumi}.}
\tightlist
\item
  Ein Spinlock kann beliebig viel CPU-Zeit verbrauchen
\item
  \mintinline{java}{yield} ist nur ein Hinweis an die JVM, es muss kein
  anderer Thread aktiviert werden
\item
  Einzig sinnvoller Anwendungsfalls: extrem ``kurze'' kritische
  Abschnitte
\end{enumerate}

\begin{minted}{java}
class SpinLock { // Avoid needing to use this
    private volatile boolean busy = false;

    synchronized void release() {
        busy = false;
    }

    void acquire() throws InterruptedException {
        int itersBeforeYield = 100; // 100 is arbitrary
        int itersBeforeSleep = 200; // 200 is arbitrary
        long sleepTime = 1; // 1msec is arbitrary
        int iters = 0;
        for(;;) {
            if(!busy) { // test-and-test-and-set
                synchronized(this) {
                    if(!busy) {
                        busy = true;
                        return;
                    }
                }
            }

            if(iters < itersBeforeYield) { // spin phase
                ++iters;
            } else if(iters < itersBeforeSleep) { // yield phase
                ++iters;
                Thread.yield();
            } else { // back-off phase
                Thread.sleep(sleepTime);
                sleepTime = 3 * sleepTime / 2 + 1; // 50 percent is arbitrary
            }
        }
    }
}
\end{minted}
